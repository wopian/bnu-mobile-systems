% !TeX root = ../index.tex
\chapter{Syntax Highlighting}

\url{https://example.org}

%\graphicspath{{1-example/images/}}
%\begin{figure}[H]
%  \caption{Output of example}
%  \centering
%  \includegraphics[width=\textwidth]{example}
%\end{figure}

You are required to complete the attached question:

Building on the in-session presentation and tests you have undertaken to date the purpose of this section of the assignment is to gain experience and consolidate your understanding and production of an ‘App’ for a mobile computing environment. Along with the development process commonly used in industry.

Mobile Computing is becoming increasingly the preferred platform used in many aspects of modern computer development environments e.g. Business, Engineering, Education, Entertainment/Media - App/Widget development etc. The module has introduced you to the mobile App Inventor from MIT and the Android Studio IDE to provide you with the capability to develop an app within a mobile context. To demonstrate your understanding of App development you can use the MIT App Inventor or the Android Studio IDE as the base for developing your choice of App for this assignment.

You need to develop an App idea, this need not be unique but might be based on an app you currently use or have seen used with a functionality adaptation.  Maybe you will consider a ‘mash-up’ of other ideas you have come across.  The main point here is to consider your idea within a market segment then design, implement, test and publish your app on a mobile device or a mobile device emulator running on a computer.

You are required to produce a brief report covering the background to the app idea covering the design, implementation, testing and publishing of your App.  The main salient aspects should be discussed e.g. various supporting stages of the development process, problems and how they were overcome, USP (unique selling point) etc. Please remember you need to justify any decisions/choices made in your design/development.

You will be asked to present your App via a short presentation, approximately 10 to 12 minutes in length.  During the presentation you may be asked questions on any aspect of the design and development of your app.

All sources used must be properly referenced and your report must be in your own words.

Expected length – Approximately1500 words supported by screen shots and possible diagrams, storyboard etc.

In all parts, you should apply the expected conventions for academic referencing.
