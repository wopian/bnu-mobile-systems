% !TeX root = ../index.tex
\chapter{App Idea}

The app idea I decided to expand upon and develop is a note taking app. I have used various apps that provide this functionality where many are either bloated in size (upwards of 80 megabytes) or clunky to use with an annoyingly long start-up with splash screens.

These apps list your notes by the last modification date with the most recent at the top. Notes can renamed, deleted and edited with some offering additional formatting (e.g Markdown WYSIWYG). Some of them offer the ability to change the app theme between light and dark and organisation of notes using folders or a tagging system.

I intend to create a light-weight alternative of these apps called \textit{My Notes} using the Android Studio IDE to make use of the Material Design system UI.

%You need to develop an App idea, this need not be unique but might be based on an app you currently use or have seen used with a functionality adaptation.  Maybe you will consider a ‘mash-up’ of other ideas you have come across.  The main point here is to consider your idea within a market segment then design, implement, test and publish your app on a mobile device or a mobile device emulator running on a computer.

%You are required to produce a brief report covering the background to the app idea covering the design, implementation, testing and publishing of your App.  The main salient aspects should be discussed e.g. various supporting stages of the development process, problems and how they were overcome, USP (unique selling point) etc. Please remember you need to justify any decisions/choices made in your design/development.
